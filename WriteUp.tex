\documentclass[12pt]{article}
\usepackage{amsmath}
\usepackage{graphicx}
\usepackage{hyperref}
\usepackage[latin1]{inputenc}

\title{The Life of a Data Scientist}
\date{\today}
\author{Revekka Kostoeva, Nikita Vemuri, Kevin Qi,Nate Young, \\Yonatan Nozik}
\begin{document}
\maketitle

\section*{I. Introduction}
The Human Activity Recognition database was built from the recordings of 30 study participants performing activities of daily living (ADL) while carrying a waist-mounted smartphone with-mounted smartphone with embedded inertial sensors. The objective was to classify activities into one of the six activities performed: walking, walking upstairs, walking downstairs, sitting, standing, and laying down. The data set was gathered from a group of 30 volunteers, within the age bracket of 19-48 years of age. The obtained dataset has been randomly partitioned into two sets, where 70% of the volunteers was selected for generating the training data and 30% the test data. 

Using Samsung Galaxy S II embedded accelerometer and gyroscope, data that was captured included 3-axial linear acceleration and 3-axial angular velocity at a constant rate of 50Hz. The sensor signals (accelerometer and gyroscope) were pre-processed by applying noise filters. The sensor acceleration signal, which has gravitational and body motion components, was separated using a Butterworth low-pass filter into body acceleration and gravity. The gravitational force is assumed to have only low frequency components, therefore a filter with 0.3 Hz cutoff frequency was used. From each window, a vector of features was obtained by calculating variables from the time and frequency domain.

Our team found this data set particularly interesting due to its possible applications in design of wearable technology, particulatrly with a focus on the medical realm. By designing an accurate model which could predict whether the readings gathered from a smartwatch were from a walking or a sitting activity, new applications are inevitable. For example, patients suffering from high blood pressure could be easily monitored by their doctors, without the blood pressure data being made vague from lack of activity information. With our model, a doctor could easily dispose of any data not recorded while the patient was sitting, and thus have a data set gathered from a single activity throughout the day from which the doctor could monitor the patient's blood pressure levels.

\section*{text}

\begin{itemize}
  \item edit the document name above by typing in the input field
  \item make changes to the body on the left and watch the preview update
  \item check the compiler output by clicking the log button
  \item format a mathematical expression like
        $\frac{1}{2\pi}\int_{-\infty}^{\infty}e^{-\frac{x^2}{2}}dx$
  \item download the document as a pdf by selecting Export $>$ Local
        Filesystem (or by clicking the desktop download button)
  \item include an image by url like this one
        \hspace*{3em}
        \includegraphics{raptor.jpg}
  \item export your work to Dropbox or Google Drive
  \item import an existing document from your local computer
  \item try using the vim or emacs keyboard shortcuts
\end{itemize}

Editing short documents online is free. View premium plans and pricing at
\url{https://latexbase.com/static/pricing} to enjoy unlimited document editing
(online or offline) and a variety of other useful features. Thanks for trying
out our service and don't hesitate to get in touch at
\href{mailto:support@latexbase.com}{support@latexbase.com}!

\end{document}
